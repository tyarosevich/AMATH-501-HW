\documentclass{article}

\usepackage{siunitx} % Provides the \SI{}{} and \si{} command for typesetting SI units
\usepackage{graphicx} % Required for the inclusion of images
\usepackage{amsmath} % Required for some math elements 
\usepackage[export]{adjustbox} % loads also graphicx
\usepackage{listings}
\usepackage{matlab-prettifier}
\usepackage{float}
\usepackage[most]{tcolorbox}
\usepackage{amsfonts}
\usepackage{color}
\usepackage{titlesec}
\usepackage{caption}
\usepackage{subcaption}
\usepackage{placeins}
\usepackage{bm}
\usepackage{esvect}
\newcommand{\uveci}{{\bm{\hat{\textnormal{\bfseries\i}}}}}
\newcommand{\uvecj}{{\bm{\hat{\textnormal{\bfseries\j}}}}}
\DeclareRobustCommand{\uvec}[1]{{%
  \ifcsname uvec#1\endcsname
     \csname uvec#1\endcsname
   \else
    \bm{\hat{\mathbf{#1}}}%
   \fi
}}


\newcommand{\R}{\mathbb{R}}

\usepackage{xcolor}

\DeclareCaptionFont{white}{\color{white}}
\DeclareCaptionFormat{listing}{%
  \parbox{\textwidth}{\colorbox{gray}{\parbox{\textwidth}{#1#2#3}}\vskip-4pt}}
\captionsetup[lstlisting]{format=listing,labelfont=white,textfont=white}
\lstset{frame=lrb,xleftmargin=\fboxsep,xrightmargin=-\fboxsep}
\titleformat{\section}[runin]
  {\normalfont\Large\bfseries}{\thesection}{1em}{}
\titleformat{\subsection}[runin]
  {\normalfont\large\bfseries}{\thesubsection}{1em}{}


\setlength\parindent{0pt} % Removes all indentation from paragraphs

\renewcommand{\labelenumi}{\alph{enumi}.} % Make numbering in the enumerate environment by letter rather than number (e.g. section 6)

%\usepackage{times} % Uncomment to use the Times New Roman font

%----------------------------------------------------------------------------------------
%	DOCUMENT INFORMATION
%----------------------------------------------------------------------------------------

\title{AMATH 501: Homework 02 \\Due October, 17 2018 \\ ID: 1064712} % Title

\author{Trent \textsc{Yarosevich}} % Author name

\date{\today} % Date for the report

\begin{document}
\maketitle % Insert the title, author and date
\setlength\parindent{1cm}

\begin{center}
\begin{tabular}{l r}
%Date Performed: December 1, 2017 \\ % Date the experiment was performed
Instructor: Mark Kot % Instructor/supervisor
\end{tabular}
\end{center}

% If you wish to include an abstract, uncomment the lines below
% \begin{abstract}
% Abstract text
% \end{abstract}

%----------------------------------------------------------------------------------------
%	SECTION 1
%----------------------------------------------------------------------------------------
\section*{\bf{1.)}}
\subsection*{a.)}
Because the gradient points in the direction of greatest increase, the negative of that vector will point in the direction of greatest decrease.
\begin{equation}
\begin{aligned}
\nabla h(x,y) = (-2x - 2)\uveci + (-2y -2)\uvecj \\
\nabla h(1, 1) = (-2(1) -2)\uveci + (-2(1) - 2)\uvecj = (-4, -4)\\ 
\|(-4, -4)\| = 4\sqrt{2} \\
\end{aligned}
\end{equation}
Using this information, we then divide the gradient at the point $(1,1)$ by its norm to create a unit vector, then multiply that by $-1$ to show the direction of steepest descent

\begin{tcolorbox}[minipage,colback=white,arc=0pt,outer arc=0pt]
\begin{equation}
-1(\frac{\nabla h(1,1)}{\|\nabla h(1,1)\|}) = -\frac{(-4, -4)}{4\sqrt{2}} = (\frac{\sqrt{2}}{2},\frac{\sqrt{2}}{2})
\end{equation}
\end{tcolorbox}
\subsection*{b.)}
Heading 'Northwest' corresponds to moving in the direction (-1,1). To find the rate of change in this direction, we dot the gradient into a unit vector in that direction to take the directional derivative.
\begin{equation}
\vv{\bm{u}} = \frac{1}{\sqrt{2}}(-1, 1)
\end{equation}
I don't know if this question is asking for the directional derivative when moving Northwest generally, or at the point (1,1), so I will provide both.
\begin{tcolorbox}[minipage,colback=white,arc=0pt,outer arc=0pt]
\begin{equation}
\begin{aligned}
\nabla h \cdot \vv{\bm{u}} = (-2x -2, -2y -2) \cdot (\frac{-1}{\sqrt{2}}, \frac{1}{\sqrt{2}}) = \frac{2x - 2y}{\sqrt{2}}\\
\nabla h(1,1) \cdot \vv{\bm{u}} = (-4, -4) \cdot (\frac{-1}{\sqrt{2}},\frac{1}{\sqrt{2}}) = \frac{4}{\sqrt{2}} - \frac{4}{\sqrt{2}} = 0
\end{aligned}
\end{equation}
\end{tcolorbox}
The result is not surprising since we can intuit that this 'mountain' is perfectly round, so at any precisely Northeasterly point (i.e. (1,1)), moving Northwest is, at that point, precisely contouring a level curve, and so there is no change in altitude. 
\subsection*{c.)}
Mark explained this to me and it made perfect sense - academic integrity and all! So we know that the top of the mountain is flat, so the directional derivative at that point, in any direction, will be zero. Since the directional derivative is the dot product of the gradient and a unit vector, i.e. a non-zero magnitude vector, we know that the gradient must be the zero vector (0,0), which is akin to saying $\frac{dh}{dx} = 0$ and $\frac{dh}{dy} = 0$, which is the approach I initially took. This gives:
\begin{equation}
\begin{aligned}
\nabla h = (-2x -2, -2y -2) = (0 , 0)\\
-2x -2 = 0\\
-2x = 2\\
x = -1\\
\text{and similarly}\\
y = -1
\end{aligned}
\end{equation}
\begin{tcolorbox}[minipage,colback=white,arc=0pt,outer arc=0pt]
Thus the highest point is at (x,y) = (-1, -1)
\end{tcolorbox}
\end{document}